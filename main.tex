\documentclass[11pt]{jarticle} 

% 絶体に必要なおまじない
% %文字以降はコメントとして無視される
% 11pt は標準の文字の大きさの指定
% 10pt とすれば小さく、12pt とすれば大きくなる

\usepackage[dvipdfmx]{graphicx} % 画像を使うためのパッケージ
\usepackage{amsmath, amsfonts} % 数式を使うためのパッケージ
\usepackage{bm} % 太字の数式を使う
\usepackage{fancyhdr}

\setlength{\oddsidemargin}{-10mm} % 紙の左のマージン
\setlength{\evensidemargin}{-10mm} % 紙の右のマージン
\setlength{\textwidth}{180mm} % テキストの幅 を指定

\setlength{\topmargin}{-15mm} % 紙の上のマージン
\setlength{\textheight}{255mm} % テキストの高さ

\lhead{東京科学大学 中間報告書} %ヘッダ左
\chead{\empty} %ヘッダ中央
\rhead{2025年8月1日提出} %ヘッダ右.コンパイルした日付を表示
\lfoot{\empty} %フッタ左
\cfoot{\thepage} %フッタ中央.ページ番号を表示
\rfoot{\empty} %フッタ右

\pagestyle{fancy}

% ここから本文
\begin{document}
\title{研究題目}
\title{比較情報を考慮した効率的な制御戦略の合成に関する研究}
\author{都竹 佑季 (22B30490) \;\; 指導教員:鄭 顕志}
\date{\empty} % 日付は自動的に今日の日付が入る
\maketitle % 上で指定した title と authorを出力
\thispagestyle{fancy}

\section{概要} % \section, \subsection, \subsubsection
離散事象システムの制御において,動作コストを最小化する制御器(制御戦略)の合成は重要である.
従来の制御戦略の合成手法は,制御戦略を合成するために中間生成物を経由しており計算空間の無駄が生じるという課題が存在する.
本研究では,この課題を解決するため,全体モデルを構築せずに制御器を合成する手法であるDirected Discrete Controller Synthesis (DDCS)\cite{ddcs}を拡張する.
具体的には,コストの比較情報を考慮した新たなヒューリスティックを導入して, 制御戦略を直接合成する手法を提案する.

\section{研究背景(既存研究)}
% 離散事象システム
イベントの発生により状態が離散的に遷移するシステムを離散事象システムと呼ぶ. 離散事象システムは状態遷移図を用いてモデル化される.

% 制御器合成
モデル化されたシステムの動作仕様や要求を元に, 実際にシステムを制御するための制御器を合成する. その際, 制御器の各動作に時間や料金, 電力などのコストを考慮することで総合的なコストを最小化する制御器の導出が重要となる. 

% 制御戦略
そこで, 比較情報をもとに制御器からコストが低くなるような動作を抽出する. 抽出して生成されたものを制御戦略と呼ぶことにする.制御戦略に基づいてシステムを動かせば, 自動的にコストが最小化された動作になる.

\section{既存研究における課題}
従来の制御戦略の合成手順\cite{eze}は以下の通りである.
\begin{enumerate}
    \item 各モデルを並列合成して,あり得る状態を全て構築した全体モデルを合成する.
    \item 全体モデルから違反状態を削減して, 安全性, 到達可能性を満たす制御器を生成する.
    \item 制御器から比較情報を元に制御戦略を抽出する.
\end{enumerate}
この方法では, 各環境モデルから制御戦略を生成するまでに中間の全体モデルを経由している. この中間生成物が計算空間の無駄となっていることが課題である.

\section{提案手法}
本研究ではDirected Descrete Controller Synthesis(DDCS)\cite{ddcs}を拡張して,比較情報を考慮した探索を行う新たなDDCSを作り解決を試みる.
DDCSは初期状態を始めとして状態を一個ずつ追加して制御器を構築していく手法である. どの状態を追加していくべきかをヒューリスティックに基づいて決定し,可能な限り早くMarking Action(Marking State)に到達できるようにする. 従来手法と異なり全体モデルを経由せず直接制御器を構築できるため必要主記憶容量を削減できる. ただし, 通常のDDCSでは比較情報が考慮されておらず全体としてコストが高い制御器が生成される可能性がある. そこで本研究で比較情報を考慮したヒューリスティックを用いた新たなDDCSを作り, 直接制御戦略を合成することを目指す.

\section{研究計画}
本研究の今後のスケジュール:\\
2025年7月-8月: ヒューリスティックを考案し実装, 実験 \\
2025年9月: 論文執筆

\appendix % Appendix はこのようにして始まる
\begin{thebibliography}{99} % 文献リスト一覧はこんな風に始める
\bibitem{eze} % bibitem でラベルを付けて
 E. Castellano, V. Braberman, N. D'Ippolito, S. Uchitel, and K. Tei, % 内容 (著者等) を記載
{\it \LaTeX : Minimising
Makespan of Discrete Controllers: A Qualitative Approach},
in 2019 IEEE 58th
Conference on Decision and Control (CDC), Nice, France, 2019, pp. 1068-1075..
\bibitem{ddcs} % bibitem でラベルを付けて
D. Ciolek, M. Duran, F. Zanollo, N. Pazos, J. Braier, V. Braberman, N. D'Ippolito, and S.
Uchitel, % 内容 (著者等) を記載
{\it \LaTeX : On-the-fly informed search of non-blocking directed controllers},
Automatica, vol. 147, No. C (2023).

\end{thebibliography} % 文献リストの最後に必要
\end{document} % 最後の締めくくり.絶体必要